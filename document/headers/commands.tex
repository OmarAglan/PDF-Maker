% Syntax Highlightling

%\DefineShortVerb[commandchars=\\\{\}]{\|}
\DefineVerbatimEnvironment{Highlighting}{Verbatim}{commandchars=\\\{\}}
% Add ',fontsize=\small' for more characters per line
\newenvironment{Shaded}{\begin{scriptsize}}{\end{scriptsize}}
\newcommand{\VerbBar}{|}
\newcommand{\VERB}{\Verb[commandchars=\\\{\}]}
\DefineVerbatimEnvironment{Highlighting}{Verbatim}{commandchars=\\\{\}}
% Add ',fontsize=\small' for more characters per line
\newcommand{\KeywordTok}[1]{\textbf{{#1}}}
\newcommand{\DataTypeTok}[1]{\underline{{#1}}}
\newcommand{\DecValTok}[1]{{#1}}
\newcommand{\BaseNTok}[1]{{#1}}
\newcommand{\FloatTok}[1]{{#1}}
\newcommand{\ConstantTok}[1]{{#1}}
\newcommand{\CharTok}[1]{{#1}}
\newcommand{\SpecialCharTok}[1]{{#1}}
\newcommand{\StringTok}[1]{{#1}}
\newcommand{\VerbatimStringTok}[1]{{#1}}
\newcommand{\SpecialStringTok}[1]{{#1}}
\newcommand{\ImportTok}[1]{{#1}}
\newcommand{\CommentTok}[1]{\textit{{#1}}}
\newcommand{\DocumentationTok}[1]{\textit{{#1}}}
\newcommand{\AnnotationTok}[1]{\textit{{#1}}}
\newcommand{\CommentVarTok}[1]{\textit{{#1}}}
\newcommand{\OtherTok}[1]{{#1}}
\newcommand{\FunctionTok}[1]{{#1}}
\newcommand{\VariableTok}[1]{{#1}}
\newcommand{\ControlFlowTok}[1]{\textbf{{#1}}}
\newcommand{\OperatorTok}[1]{{#1}}
\newcommand{\BuiltInTok}[1]{{#1}}
\newcommand{\ExtensionTok}[1]{{#1}}
\newcommand{\PreprocessorTok}[1]{\textbf{{#1}}}
\newcommand{\AttributeTok}[1]{{#1}}
\newcommand{\RegionMarkerTok}[1]{{#1}}
\newcommand{\InformationTok}[1]{\textit{{#1}}}
\newcommand{\WarningTok}[1]{\textit{{#1}}}
\newcommand{\AlertTok}[1]{\textbf{{#1}}}
\newcommand{\ErrorTok}[1]{\textbf{{#1}}}
\newcommand{\NormalTok}[1]{{#1}}





\newcommand{\myfigurewithoutcaption}[1]{{\bfseries \myfigurebabel{ }#1}}
\newcommand{\myfigurewithcaption}[2]{{\bfseries \myfigurebabel{ }#1{\quad}}#2}
\renewcommand{\iff}{\Longleftrightarrow}

% Definition der Fussnoten
% ------------------------
%\KOMAoptions{footnotes=multiple}


\DeclareTextSymbol{\textlongs}{TS1}{115} 

\deffootnote[2.2em]{2.2em}{0em}{\makebox[2.2em][l]{\thefootnotemark}}

\newcommand{\badchar}[1]
{\textbf{?}}


\newcommand{\myplainurl}[1]
{{\ttfamily  \url{#1}}}


\newcommand{\myfnhref}[2]
{{#2} \^{}{\{\ttfamily  \url{#1}\}} }

\newcommand{\mymchref}[2]
{}


\newcommand{\mytabhref}[2]
{{#2}\protect\footnote{\ttfamily \url{#1} }}
%{\textsc{#2}}


\newcommand{\myfnlref}[2]
{{#2} \^{}\{\mychapterbabel \ref{#1} \mypagebabel {$\text{}$} \pageref{#1}\}}

\newlength{\fnwidth}
\setlength{\fnwidth}{\linewidth}
\addtolength{\fnwidth}{-10mm}

\newcommand{\myhref}[2]
{{#2}\protect\footnote{    \begin{minipage}{\fnwidth} \ttfamily \url{#1}  \end{minipage}}} 

\newcommand{\mylref}[2]
{{#2}\protect\footnote{\mychapterbabel {$\text{}$} \ref{#1} \mypagebabel {$\text{}$} \pageref{#1}}}

\newcommand{\myfnsref}[2]
{\text{#2} \^{}\{\text{#1} \}}

\newcommand{\mysref}[2]
{\text{#2}\protect\footnote{#1}}

\newcommand{\TickYes}{\checkmark}


% Kompatibilität, damit myfootnote nichts ins Leere läuft
\newcommand{\myfootnote}[1]
%{\footnote{\quad{}#1}}
{\footnote{#1}}


% Auflistungen
% ------------
% Standardvorschlag für itemize
%\newenvironment{myitemize}{\begin{itemize}}{\end{itemize}}
%\newenvironment{myenumerate}{\begin{enumerate}}{\end{enumerate}}
\newenvironment{myquote}{\begin{itemize}[{}]}{\end{itemize}}
\newenvironment{myblockquote}{\begin{itemize}[{\quad}]}{\end{itemize}}

\newenvironment{mydescription}{

\begin{inparablank}}{\end{inparablank}} 
% Alternativen ohne Einrückung
\newenvironment{myitemize}{\begin{compactitem}[\textbullet]}{\end{compactitem}}
\newenvironment{myenumerate}{\begin{compactenum}}{\end{compactenum}}

% einige weitere Festlegungen
% ---------------------------
% \breakslash is used for URLs to allow linebreaking
\newcommand{\mybreakslash}{\discretionary{/}{}{/}}

\newlength{\mylength}
\newlength{\myhight}
\newlength{\myshadingheight}
\newcommand{\myoverline}[1]
{\settowidth{\mylength}{#1} \settoheight{\myhight}{#1}
\makebox[-3pt][l]{#1}
\rule[\myhight+1pt]{\mylength}{0.15mm}}

% Teile von Büchern
\newcommand{\mypart}[1]
%{\part{#1}}
{\addtocontents{toc}{\protect\vspace{7.5mm} \textbf{\Large {#1}}}}

% minitoc vorbereiten, aber standardmäßig unterdrücken
\newcommand{\myminitoc}{}

% Haupttitel
% ----------
%\newcommand{\mymaintitle}[1]
%{\definecolor{shadecolor}{gray}{0.9}\begin{shaded}
%\begin{center}
%\Huge \bfseries 
%#1 
%\end{center}
%\end{shaded}}

%\newcommand{\mysubtitle}[1]
%{\begin{center}
%\LARGE \bfseries 
%#1
%\end{center}}

\newcommand{\mysubtitle}[1]{\subtitle{#1}}
\newcommand{\mymaintitle}[1]{\title{#1}}
\newcommand{\myauthor}[1]{\author{#1}}


% this is for getting rid of a lintian complaint about
% the German translation of the English word resolution which
% I can not represent here literally according to lintian
\newcommand{\resdhunlongstring}[0]{Res}
\newcommand{\sourcedhunlongstring}[0]{source}
\newcommand{\redshunlongstringsource}[0]{\resdhunlongstring\sourcedhunlongstring}


% Metadaten
% ---------
\newcommand{\fetchurlcaption}[0]
{\mysref{In den Metadaten erläutert unter: {\itshape Adresse der elektronischen \redshunlongstringsource zur Abholung (O)}.}{URL zur Abholung}}

\newcommand{\bookcaption}[0]
{\mysref{In den Metadaten erläutert unter: {\itshape Adresse der elektronischen \redshunlongstringsource (O)}.}{Buch (Hauptseite)}}

\newcommand{\functionalgroupcaption}[0]
{\mysref{In den Metadaten erläutert unter: {\itshape Angaben zum Inhalt: DDC-Sachgruppe der Deutschen Nationalbibliografie oder Warengruppen-Systematik des Deutschen Buchhandels (O)}.}{Sachgruppe(n)} }

\newcommand{\futhertopicscaption}[0]
{\mysref{In den Metadaten erläutert unter: {\itshape Angaben zum Inhalt: weitere Klassifikationen / Thesauri (F)}.}{Weitere Themen}}

\newcommand{\mainauthorscaption}[0]
{Hauptautor(en)}

\newcommand{\projecttexniciancaption}[0]
{Betreuer}

\newcommand{\organizationscaptions}[0]
{\mysref{In den Metadaten erläutert unter: {\itshape Beteiligte Organisationen (F)}.}{Organisation(en)}}

\newcommand{\datecaption}[0]
{Erscheinungsdatum}

\newcommand{\issuecaption}[0]
{Ausgabebezeichnung}

\newcommand{\standardcodecaption}[0]
{Standardnummer }

\newcommand{\maintitlecaption}[0]
{Haupttitel}

\newcommand{\publishercaption}[0]
{\mysref{In den Metadaten erläutert unter: {\itshape Verlag / Verlegende Stelle (O)}.}{Verlegende Stelle} }

\newcommand{\publishercitycaption}[0]
{Verlagsort}

\newcommand{\shelfcaption}[0]
{Wikibooks-Regal}

\newcommand{\sizecaption}[0]
{Umfang}


\newcommand{\Alpha}{\mathrm{A}}
\newcommand{\Beta}{\mathrm{B}}
\newcommand{\Epsilon}{\mathrm{E}}
\newcommand{\Zeta}{\mathrm{Z}}
\newcommand{\Eta}{\mathrm{H}}
\newcommand{\Iota}{\mathrm{I}}
\newcommand{\Kappa}{\mathrm{K}}
\newcommand{\Mu}{\mathrm{M}}
\newcommand{\Nu}{\mathrm{N}}
\newcommand{\Rho}{\mathrm{P}}
\newcommand{\Tau}{\mathrm{T}}
\newcommand{\Chi}{\mathrm{X}}






